%%%%%%%%%%%%%%%%%%%%%%%%%%%%%%%%%%%%%%%%%%%%%%%%%%%%%%%%%%%%%%%%%%%%
%% I, the copyright holder of this work, release this work into the
%% public domain. This applies worldwide. In some countries this may
%% not be legally possible; if so: I grant anyone the right to use
%% this work for any purpose, without any conditions, unless such
%% conditions are required by law.
%%%%%%%%%%%%%%%%%%%%%%%%%%%%%%%%%%%%%%%%%%%%%%%%%%%%%%%%%%%%%%%%%%%%

\PassOptionsToPackage{
  plainpages = false,               								
  pdfpagelabels,
  unicode,  
  colorlinks = true,
  linkcolor = link-grey,	% PC version
  citecolor = crimson,		% PC version
  urlcolor = blue,			% PC version
  % linkcolor = black,		% Print version
  % citecolor = black,		% Print version
  % urlcolor = black,		% Print version
}{hyperref}

\documentclass[
  digital,  	% PC version
  color,		% PC version
  oneside,   	% PC version
  % printed 	% Printed version
  % monochrome	% Printed version
  % twoside		% Printed version
  12pt,
  nocover,
  table,
  nolof,
  nolot,
  %% More options are listed in the user guide at
  %% <http://mirrors.ctan.org/macros/latex/contrib/fithesis/guide/mu/fi.pdf>.
]{fithesis3}

%% The following section sets up the locales used in the thesis.
\usepackage[resetfonts]{cmap} %% We need to load the T2A font encoding
\usepackage[T1,T2A]{fontenc}  %% to use the Cyrillic fonts with Russian texts.
\usepackage[
  main=english,
  english, german, russian, czech, slovak
]{babel}
%% fonts:
\usepackage{paratype}
\def\textrussian#1{{\usefont{T2A}{PTSerif-TLF}{m}{rm}#1}}

%%
%% The following section sets up the metadata of the thesis.
\thesissetup{
    date          = \the\year/\the\month/\the\day,
    university    = mu,
    faculty       = fi,
    type          = mgr,
    author        = Ľubomír Obrátil,
    gender        = m,
    advisor       = {RNDr. Petr Švenda, Ph.D.},
    title         = {Randomness Testing Toolkit},
    TeXtitle      = {WORK TITLE: The automated testing of randomness with multiple statistical batteries},
    keywords      = {TODO},
    TeXkeywords   = {TODO},
    abstract      = {TODO},
    thanks        = {TODO},
    bib           = bibliography.bib,
}
\usepackage{makeidx}      %% The `makeidx` package contains
\makeindex                %% helper commands for index typesetting.
\thesisload

\newcommand{\sidemargin}{3.6cm}

\usepackage[top=3cm, bottom=3.5cm, left=\sidemargin, right=\sidemargin]{geometry}  		% PC version
%\usepackage[top=3cm, bottom=3.5cm, left=4.6cm, right=2.6cm]{geometry}		% Printed version

%% These additional packages are used within the document:
\usepackage{paralist} %% Compact list environments
\usepackage{amsmath}  %% Mathematics
\usepackage{amsthm}
\usepackage{url}
\usepackage{amsfonts}
\usepackage{markdown} %% Lightweight markup

\usepackage{listings} %% Source code highlighting
\lstset{
  	basicstyle      = \ttfamily,%
  	identifierstyle = \color{black},%
  	keywordstyle    = \color{blue},%
  	keywordstyle    = {[2]\color{cyan}},%
  	keywordstyle    = {[3]\color{olive}},%
 	stringstyle     = \color{teal},%
	commentstyle    = \itshape\color{magenta}
}

\usepackage{xcolor}
\definecolor{link-grey}{rgb}{0.3,0.3,0.3}
\definecolor{code-bg-grey}{rgb}{0.85,0.85,0.85}
\definecolor{code-bg-light_grey}{rgb}{0.92,0.92,0.92}
\definecolor{crimson}{rgb}{0.6,0,0}


%%%%%%%%%%%%%%%%%%%%%%%%%%%%%%%%%%%%%%%%%%%%%%%%%%%%%%%%%%%%%%%%%%%%%%%%%%%%%%%%%%%%%%%%
%%%%%%%%%%%%%%%%%%%%%%%%%%%%%%%%%%%%%%%%%%%%%%%%%%%%%%%%%%%%%%%%%%%%%%%%%%%%%%%%%%%%%%%%
% =================================== TEXT BEGINNING =================================== 
%%%%%%%%%%%%%%%%%%%%%%%%%%%%%%%%%%%%%%%%%%%%%%%%%%%%%%%%%%%%%%%%%%%%%%%%%%%%%%%%%%%%%%%%
%%%%%%%%%%%%%%%%%%%%%%%%%%%%%%%%%%%%%%%%%%%%%%%%%%%%%%%%%%%%%%%%%%%%%%%%%%%%%%%%%%%%%%%%
\begin{document}

\chapter{Introduction}
\begin{itemize}
\item Randomness, why should we test it (defects, low entropy, etc...)
\item Statistical testing of randomness
\end{itemize}

\chapter{Overview of statistical batteries}
\begin{itemize}
\item Terminology related to batteries - battery, test, variant of a test, subtest, statistics, p-values
\item nist sts, dieharder, testu01
\item Each battery: overview, tests, tests into (subtests, default parameters, variants in default run), known defectes
\end{itemize}

\chapter{Randomness Testing Toolkit}
\begin{itemize}
\item Motivation - unified interface to batteries, ease of use, unified result format/representation
\item Local execution of RTT - battery and toolkit configuration, installation, brief implementation and interface overview - more thorough in documentation and comments
\item Local result format - either database or file output storage
\item Remote execution of RTT - toolkit deployed on server infrastructure, system overview (database, frontend, backend(s), storage), accessible through ssh on limited system or via web interface (django), results in database
\item Remote results of RTT - email notification, webpage layout
\end{itemize}

\chapter{Interpretation of results of RTT}
\begin{itemize}
\item Grouping subtests together - eliminating intertest bias
\item How grouping works - theory, Sidak correction, partial p-value, fail/pass of a test
\end{itemize}
 
\chapter{Analysis of outputs of cryptographic functions, comparison with EACirc}
\begin{itemize}
\item How the data were tested
\item List functions
\item List interesting (differing) results - Dieharder, NIST STS, TestU01, EACirc, polynomials(???)
\end{itemize}

\chapter{Analysis of DIEHARDER results on quantum random data}
\begin{itemize}
\item Statistical intro, uniformity, first vs. second level p-value, etc...
\item Two experiments - continuous p-values, blocks of 2nd level
\item Results - non-uniform, where it will begin to show on 2nd level results
\end{itemize}

\chapter{Conclusions}
\begin{itemize}
\item Developed user-friendly tool for easy analysis of arbitrary binary data - Randomness Testing Toolkit
\item Interpretation of results
\item Comparison of batteries with EACirc, polynomials
\item Defects in Dieharder, their relevance, etc...
\item Future work, same analysis on TestU01, dependence between tests(?), continuous development of RTT, call for flawless statistical battery ( :) )
\end{itemize}

\appendix

\printbibliography

\chapter{An appendix}
\begin{huge}
TODO
\end{huge}

\end{document}
